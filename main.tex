\documentclass[a4paper,12pt,oneside]{extbook}
\usepackage[english,russian]{babel}
\usepackage{fontspec}
\usepackage{subcaption}
\usepackage{graphicx}
\usepackage{indentfirst}
\usepackage{caption}
\usepackage{wrapfig}
\usepackage{xcolor,soul,lipsum}
\usepackage{amsmath}
\usepackage{amsthm}
\usepackage{hyperref}
\usepackage{enumitem} % no item sep in list
\usepackage[explicit]{titlesec}
\usepackage{amssymb}
\usepackage{titletoc}
\usepackage{tocvsec2}
\usepackage{tocloft}
\usepackage[b]{esvect}
\usepackage{mdframed}
\usepackage{textcomp}
\usepackage{multicol}
\usepackage[%
    left=0.8in,%
    right=0.8in,%
    top=0.8in,%
    bottom=1in,%
]{geometry}%

\DeclareMathOperator{\sign}{sign}

\newmdenv[
    linewidth=2pt,
    align=center,
    topline=false,
    bottomline=false,
    rightline=false,
    skipabove=\topsep,
    skipbelow=\topsep,
]{siderules}

\providecommand{\pgfsyspdfmark}[3]{}

\newcommand{\newpar}{$ $\par\nobreak\ignorespaces}
\newcommand{\makeline}{\noindent\makebox[\linewidth]{\rule{0.8\paperwidth}{0.4pt}}}

\newenvironment{breakenv}[3][]{\noindent\textbf{#1}#2#3.\newpar}{\bigskip}

\newtheoremstyle{numbered}{}{}{}{}{}{}{.5em}{\textbf{#1\;#2.}\;#3}

\newtheoremstyle{unnumbered}{}{}{}{}{}{}{.5em}{\textbf{#1\if\relax\detokenize{#3}\relax\else\;#3\newpar\fi.}\par}

\newenvironment{definition}[1][]{\noindent\textbf{Определение.\if\relax\detokenize{#1}\relax\else\;#1.\fi}\newpar}{}

\theoremstyle{numbered}
\newtheorem{property}{Свойство}[section]
\renewcommand{\theproperty}{\arabic{property}}

\newtheoremstyle{named}{}{}{}{}{\bfseries}{}{.5em}{
    #1\if\relax\detokenize{#2}\relax.\else\;#2.\fi
    \if\relax\detokenize{#3}\relax
    \else
        \;#3.
    \fi
}

\theoremstyle{unnumbered}
\newtheorem*{theorem*}{Теорема}

\theoremstyle{named}
\newtheorem{theorem}{Теорема}[section]
\renewcommand{\thetheorem}{\arabic{theorem}}

\theoremstyle{unnumbered}
\newtheorem*{lemma*}{Лемма}

\theoremstyle{named}
\newtheorem{lemma}{Лемма}[section]
\renewcommand{\thelemma}{\arabic{lemma}}

\theoremstyle{named}
\newtheorem*{consequence}{Следствие}

\theoremstyle{named}
\newtheorem*{note}{Замечание}

\renewenvironment{proof}[1][]{\breakenv[Доказательство]{\if\relax\detokenize{#1}\relax\else\;\fi}{\textbf{#1}}}{\smallskip\newpar \hfill\textit{Что и требовалось доказать.}}

\renewcommand\qedsymbol{}

\pagestyle{plain}
\setmainfont{PT Serif}

\titleformat{\section}
{\Large}{\textbf{\thesection.}}{0.5em}{\textbf{#1}}

\titleformat{\chapter}
{\Huge}{\textbf{\chaptername\ \thechapter.}}{0.5em}{\textbf{#1}}

\titlecontents{chapter}% <section-type>
[0pt]% <left>
{\vspace{0.5cm}}% <above-code>
{\bfseries\chaptername\ \thecontentslabel.\ }% <numbered-entry-format>
{}% <numberless-entry-format>
{\bfseries\hfill\contentspage}

\renewcommand{\cftsecfont}{\mdseries}
\renewcommand{\cftsecpagefont}{\mdseries}

\newcommand{\overbar}[1]{\mkern 1.5mu\overline{\mkern0mu#1\mkern-1.5mu}\mkern 1.5mu}

\hypersetup{
    colorlinks=true,
    linkcolor=blue,
    filecolor=magenta,
    urlcolor=cyan,
    pdftitle={Ответы на билеты по математике 2021},
    pdfpagemode=FullScreen,
}

\newcommand{\plink}[2]{\hyperref[#1]{\color{blue}\underline{#2}}}

\captionsetup[figure]{labelformat=empty, labelsep=none}
\graphicspath{ {./images/} }

\title{
    Конспекты по математике \\
    \vspace{2cm} 2 семестр \\
    \vspace{2cm} ИКТ \\
    2021 — 2022
    \vfill
}
\author{
    Автор: \\
    Даниил Швалов
}
\date{}

\setlength{\cftbeforesecskip}{6pt}


\begin{document}

\begin{titlepage}
    \pagestyle{empty}
    \cleardoublepage
    \maketitle
    \thispagestyle{empty}
\end{titlepage}

\setcounter{page}{2}
{
    \setcounter{tocdepth}{4}
    \hypersetup{linkcolor=black}
    \tableofcontents
}

\chapter{Интегралы}%
\label{cha:Интегралы}

\section{Неопределенный интеграл}%
\label{sec:Неопределенный интеграл}

\subsection{Непосредственное интегрирование}%
\label{sub:Непосредственное интегрирование}

\begin{definition}
    Функция \(F(x)\) называется первообразной для функции \(f(x)\), если выполняется следующее:
    \begin{gather*}
        F'(x) = f(x) \text{ или } dF(x) = f(x) dx
    \end{gather*}

    Если функция \(f(x)\) имеет первообразную \(F(x)\), то она имеет \textbf{бесконечное множество первообразных}, причем все они содержатся в выражении:
    \begin{gather*}
        F(x) + C, \quad C = const
    \end{gather*}

    \textbf{Неопределенным интегралом} от функции \(f(x)\) (или от выражения \(f(x)dx\)) называется совокупность всех ее первообразных:
    \begin{gather*}
        \int f(x)dx = F(x) + C
    \end{gather*}
\end{definition}

\subsection{Свойства неопределенного интеграла}%
\label{sub:Свойства неопределенного интеграла}
\begin{enumerate}
    \item {\((\int f(x) dx)' = f(x)\)}
    \item {\(d(\int f(x) dx) = f(x)dx\)}
    \item {\(\int d F(x) = F(x) + C\)}
    \item {\(\int a f(x) dx = a \int f(x) dx, a = const\)}
    \item {\(\int [f(x) \pm g(x)]dx = \int f(x)dx \pm \int g(x)dx\)}
    \item {
          \(
          \begin{cases}
              \int f(x) dx = F(x) + C \\
              u = \varphi(x)
          \end{cases}
          \implies
          \int f(u) du = F(u) + C
          \)
          }
\end{enumerate}

\subsection{Полезные тригонометрические тождества}%
\label{ssub:Тригонометрические тождества}

\begin{multicols}{2}
    \begin{enumerate}
        \item {\(\sin^{x} + \cos^2{x} = 1\)}
        \item {\(\tg{x} \cdot \ctg{x} = 1\)}
        \item {\(1 + \tg^2{x} = \dfrac{1}{\cos^2{x}} = \sec^2{x}\)}
        \item {\(1 + \ctg^2{x} = \dfrac{1}{\sin^2{x}} = \cosec^2{x}\)}
    \end{enumerate}
\end{multicols}

\subsection{Таблица основных интегралов}%
\label{sub:Таблица основных интегралов}

\begin{multicols}{2}
    \begin{enumerate}
        \item {\(\int dx = x + C\)}
        \item {\(\int x^m dx = \dfrac{x^{m + 1}}{m + 1} + C, m \neq -1\)}
        \item {\(\int \dfrac{dx}{x} = \ln|x| + C\)}
        \item {\(\int \dfrac{dx}{1 + x^2} = \arctg{x} + C\)}
        \item {\(\int \dfrac{dx}{\sqrt{1 - x^2}} = \arcsin{x} + C\)}
        \item {\(\int e^x dx = e^x + C\)}
        \item {\(\int a^x dx = \dfrac{a^x}{\ln{a}} + C\)}
        \item {\(\int \sin{x}dx = -\cos{x} + C\)}
        \item {\(\int \cos{x}dx = \sin{x} + C\)}
        \item {\(\int \sec^2{x}dx = \tg{x} + C\)}
        \item {\(\int \cosec^2{x}dx = -\ctg{x} + C\)}
        \item {\(\int \sh{x}dx = \ch{x} + C\)}
        \item {\(\int \ch{x}dx = \sh{x} + C\)}
        \item {\(\int \dfrac{dx}{\ch^2{x}} = \th{x} + C\)}
        \item {\(\int \dfrac{dx}{\sh^2{x}} = -\cth{x} + C\)}
        \item {\(\int \dfrac{f'(x)}{f(x)}dx = \ln|f(x)| + C\)}
        \item {\(\dfrac{f'(x)}{\sqrt{f(x)}}dx = 2\sqrt{f(x)} + C\)}
        \item {\(\int \dfrac{dx}{x^2 + a^2} = \dfrac{1}{a} \arctg{\dfrac{x}{a}} + C\)}
        \item {\(\int \dfrac{dx}{x^2 - a^2} = \dfrac{1}{2a} \ln {\Big|\dfrac{x - a}{x + a}\Big|} + C\)}
        \item {\(\int \dfrac{dx}{\sqrt{a^2 - x^2}} = \arcsin{\dfrac{x}{a}} + C\)}
        \item {\(\int \dfrac{dx}{\sqrt{x^2 + \lambda}} = \ln{|x + \sqrt{x^2 + \lambda}|} + C\)}
        \item {\(\int \dfrac{dx}{\sin{x}} = \ln \Big|\tg{\dfrac{x}{2}}\Big| + C\)}
        \item {\(\int \dfrac{dx}{\cos{x}} = \ln{\Big|\tg(\dfrac{x}{2} + \dfrac{\pi}{4})\Big|} + C\)}
        \item {\(\int \tg{x}dx = -\ln|\cos{x}| + C\)}
        \item {\(\int \ctg{x}dx = \ln|\sin{x}| + C\)}
    \end{enumerate}
\end{multicols}


\subsection{Замена переменной в неопределенном интеграле}%
\label{sub:Замена переменной в неопределенном интеграле}

Замена переменной производится с помощью подстановок 2-х типов:
\begin{enumerate}
    \item {
          Если \(x = \varphi(t)\), где \(\varphi(t)\) — монотонная, непрерывно дифференцируемая функция новой переменной \(t\), то формула замены переменной:
          \begin{gather*}
              \int f(x)dx = \int f [\varphi(t)] \cdot \varphi'(t)dt
          \end{gather*}
          }
    \item {
          Если \(u = \psi(x)\), где \(u\) — новая переменная, то формула замены переменной:
          \begin{gather*}
              \int f[\psi(x)] \cdot \psi'(x) dx = \int f(u)du
          \end{gather*}
          }
\end{enumerate}

\begin{gather*}
    \int f(ax + b) dx = \frac{1}{a} F(ax + b) + C
\end{gather*}

где \(f\) — табличная функция, а \(F\) — первообразная для \(f\).

\subsection{Интегрирование по частям}%
\label{sub:Интегрирование по частям}

Пусть \(u = \varphi(x)\), \(v = \psi(x)\) — непрерывно дифференцируемы от \(x\). Тогда:
\begin{gather*}
    \int udv = u \cdot v - \int vdu
\end{gather*}

\textbf{Целесообразность}. В качестве \(u\) берется такая функция, которая упрощается при дифференцировании. В качестве \(dv\) берется такая функция, что интеграл ее либо известен, либо может быть найден. Например:

\begin{enumerate}
    \item {Если \(\int P(x) e^{ax} dx\), \(\int P(x) \sin(ax)dx\), \(\int P(x) \cos(ax)dx\), \(\int P(x)\) — многочлен, то в качестве \(u\) используется \(P(x)\), а в качестве \(dv\) — \(e^{ax}dx\), \(\sin(ax)dx\), \(\cos(ax)dx\) соответственно.}
    \item {Если \(\int P(x) \ln{x}dx\), \(\int P(x) \arcsin{x}dx\), \(\int P(x) \arccos{x}dx\) — многочлен, то в качестве \(u\) используется \(\ln{x}\), \(\arcsin{x}\), \(\arccos{x}\) соответственно, а в качестве \(dv\) — многочлен \(P(x)dx\).}
\end{enumerate}

\end{document}
